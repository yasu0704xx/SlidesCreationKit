\documentclass[xcolor=svgnames,dvipdfmx,cjk]{beamer} 
\AtBeginDvi{\special{pdf:tounicode 90ms-RKSJ-UCS2}} 
\usetheme[progressbar=frametitle]{metropolis}
\setbeamertemplate{headline}[miniframes]
\setbeamercolor{background canvas}{bg=Snow} 
\setbeamercolor{frametitle}{fg=white, bg=gray!40!black}  
\setbeamercolor{section in head/foot}{fg=pink!70!red, bg=gray!30!black}
\setbeamercolor{alerted text}{fg=orange!50!red} 
\usefonttheme{professionalfonts}
\setbeamertemplate{theorems}[numbered]
\newtheorem{thm}{Theorem}[section]
\newtheorem{proposition}[thm]{Proposition}
\theoremstyle{example}
\newtheorem{exam}[thm]{Example}
\newtheorem{remark}[thm]{Remark}
\newtheorem{question}[thm]{Question}
\newtheorem{prob}[thm]{Problem}
\def\Avar{\text{Avar}}
\def\var{\text{var}}
\def\Var{\text{Var}}
\def\cov{\text{cov}}
\def\Cov{\text{Cov}}
\def\E{\mathbb{E}}
\def\R{\mathbb{R}}
\def\P{\mathbb{P}}
\def\parrow{\xrightarrow{p}}
\def\darrow{\xrightarrow{d}}
\def\plim{\text{plim }}
\usepackage{bbm}
\usepackage{ascmac}
\usepackage{tcolorbox}

%%%%%%ナビゲーションバー%%%%%%%%%%%%%%%%%%%%%%%%%%%%%%%%%%%%%%%%%%%%
%\useoutertheme{miniframes}
%\setbeamertemplate{headline}[miniframes theme]
%\setbeamertemplate{mini frames}[circle]

%\setbeamertemplate{headline}
%{%
%  \begin{beamercolorbox}{section in head/foot}
%    \vskip2pt\insertnavigation{\paperwidth}\vskip2pt
%  \end{beamercolorbox}%
%}
%%%%%%%%%%%%%%%%%%%%%%%%%%%%%%%%%%%%%%%%%%%%%%%%%%%%%%%%%%%%%%%%%%%

\begin{document} 

%%%%%講演に関する情報%%%%%%%%%%%%%%%%%%%%%%%%%%%%%%%%%%%%%%%%
\title{Censored Models} 
\subtitle{Li and Racine (2007, Chapter 11)}
\author{Yasuyuki Matsumura}
\institute{Graduate School of Economics, Kyoto University}
\date{\today} % 日付を自動で挿入

%%%%%タイトルページ%%%%%%%%%%%%%%%%%%%%%%%%%%%%%%%%%%%%%%%%%%
\begin{frame}                  
\titlepage            
\end{frame}


%%%%%%%%%%%%%%%%%%%%%%%%%%%%%%%%%%%%%%%%%%%%%%%%%%%%%%%%%%%%%%%
%%%%%%%%%%%%%%%%%%%%%%%%%%%%%%%%%%%%%%%%%%%%%%%%%%%%%%%%%%%%%%%
%%%%%%%%%%%%%%%%%%%%%%%%%%%%%%%%%%%%%%%%%%%%%%%%%%%%%%%%%%%%%%%

%Before showwing the contents of the slides...



\begin{frame}{References}
  \begin{itemize}
    \item Textbooks and Lecture Notes:
    \begin{itemize}
      \item Horowitz, J. L. (2009)
            \textit{Semiparametric and Nonparametric Methods in Econometrics,}
            Springer.
      \item Li, Q., and J. S. Racine (2007). 
            \textit{Nonparametric Econometrics: Theory and Practice,} 
            Princeton University Press.
      \item Serfling, R. J. (1980)
            \textit{Approximation Theorems of Mathematical statistics.} Wiley.
      \item van der Vaart, A. W. (2000)
            \textit{Asymptotic Statistics.} Cambridge.
      \item 末石直也 (2024) 『データ駆動型回帰分析:計量経済学と機械学習の融合』日本評論社.
      \item 西山慶彦,人見光太郎 (2023) 『ノン・セミパラメトリック統計解析(理論統計学教程:数理統計の枠組み)』共立出版.
      \item ECON 718 NonParametric Econometrics (Bruce Hansen, Spring 2009, University of Wisconsin-Madison).
      \item セミノンパラメトリック計量分析(末石直也,2014年度後期,京都大学).
    \end{itemize}
  \end{itemize}
\end{frame}




\end{document}