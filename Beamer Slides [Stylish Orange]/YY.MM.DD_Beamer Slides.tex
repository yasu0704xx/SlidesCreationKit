\documentclass[xcolor=svgnames,dvipdfmx,cjk]{beamer} 
\AtBeginDvi{\special{pdf:tounicode 90ms-RKSJ-UCS2}} 
\usetheme{metropolis}
\usefonttheme{professionalfonts}
\setbeamertemplate{theorems}[numbered]
\newtheorem{thm}{Theorem}[section]
\newtheorem{proposition}[thm]{Proposition}
\theoremstyle{example}
\newtheorem{exam}[thm]{Example}
\newtheorem{remark}[thm]{Remark}
\newtheorem{question}[thm]{Question}
\newtheorem{prob}[thm]{Problem}
\usepackage{bbm}
\usepackage{ascmac}


\begin{document} 

%%%%%講演に関する情報%%%%%%%%%%%%%%%%%%%%%%%%%%%%%%%%%%%%%%%%
\title[足元表示のタイトル]{トップページのタイトル} 
\author[Y. Matsumura]{Yasuyuki Matsumura}          
\institute[]{Graduate School of Economics, Kyoto University} 
\date{\today}


%%%%%タイトルページ%%%%%%%%%%%%%%%%%%%%%%%%%%%%%%%%%%%%%%%%%%
\begin{frame}                  
\titlepage                     
\end{frame}


%%%%%目次のページ%%%%%%%%%%%%%%%%%%%%%%%%%%%%%%%%%%%%%%%%%%%%
%\begin{frame}{Agenda}                  
%\tableofcontents
%\end{frame}
%本文中に挿入するSECTION環境によって定義された目次が自動で反映される
%いらない気もする

\section{セクション1}

\begin{frame}{このページのタイトル}
  本文
  \begin{itemize}
    \item 箇条書き
    \item 箇条書きの中に
    \begin{itemize}
      \item さらに箇条書き
      \item もできます
    \end{itemize}
    \item 色を変えたいときは\alert{alert}コマンドを使います.
  \end{itemize}
\end{frame}

\section{References}


\begin{frame}{References}
  たとえば,以下のように書きます.
  \begin{itemize}
    \item Li, Q. and J. S. Racine, (2007). 
          \textit{Nonparametric Econometrics: Theory and Practice,} 
          Princeton University Press.
    \item 末石直也 (2024) 『データ駆動型回帰分析:計量経済学と機械学習の融合』日本評論社.
    \item 西山慶彦,人見光太郎 (2023) 『ノン・セミパラメトリック統計解析(理論統計学教程:数理統計の枠組み)』共立出版.
  \end{itemize}
\end{frame}






























\end{document}
